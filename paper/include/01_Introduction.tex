\section{Introduction}


The aim of structural bioinformatics is the analysis and targeted manipulation of three dimensional structures of biological macromolecules such as proteins and nucleic acids (RNA/DNA). This research area combines and applies knowledge from diverse disciplines ranging from fundamental physical laws such as Newton's equation of motion to complex situations requiring in-depth knowledge of biochemistry and advanced numerical computing methodologies. More precisely, molecular modeling techniques typically rely on molecular mechanics in which a molecular force fields is being used to compute the energy of the examining structure.  
Resulting applications include structure minimization, molecular dynamics (MD) simulations and molecular docking scenarios. 
The importance of these molecular applications has been highlighted during the Covid-19 pandemic. Molecular modeling techniques and in particular docking suites were attracting widespread interest: First, the structure of virus proteins were predicted and examined including analysis of the effects of mutations originating from various virus strains. The knowledge of these molecular functions was then used in the context of rational drug design for the purpose of finding a potential drug for the treatment of Covid-19. \\

Another vital aspect of this research field has been demonstrated through the Covid-19 pandemic, which is the availability of molecular structure data. The availability of experimentally resolved structures was limited for many years leading to a slowdown of the progress in this area.  In recent years, the number of experimentally structures is increasing dramatically through advances in electron microscopy and with the rapid rise of computed structures, the availability no longer restrain the development of structural bioinformatics applications.  \\
Software development in interdisciplinary research areas was, and still is, typically challenging. Software packages for handling molecular structures and molecular applications are available for many years and can be divided into open-source and closed-sourced tools. One example for the latter are the tools provided by schroedinger:\texttt{ProteinPreparationWizard} deals with the preprocessing of protein structural models including missing atoms, missing hydrogens, bond order assignment etc. or \texttt{LiveDesign} for the docking and design of ligands. Although Schroedinger provides entire platforms for several task, these have the indisputable disadvantage of being closed-source. 
Most open-source software packages were created roughly from 1995 to 2010 and were often only designed for one specific task e.g., implementation of a structure minimization algorithm, docking algorithm... \todo{cite single approaches} \\
An exception to this was the introduction of \ballFull (\ball) by Kohlbacher et al. in 1996.  \ball is a well designed framework for molecular structure analysis providing a rich functionality namely file import and export of most common file formats, structure preprocessing, molecular mechanics, advanced solvation methods and visualization options. \ball used to have one of the biggest user communities in this field as can be assumed from the long lasting maintenance and ongoing development. In 2010 a new version was published featuring Python bindings for rapid application development (RAD). \\
Like \ball, more recent software packages were also written in C++ with additional Python bindings \cite{Doerr2016HTMD}. While C++ is necessary for the required efficiency of the programms, it effectively hinders the rapid prototyping of molecular algorithms.  \\

While the development of software packages for structural bioinformatics remains a challenging task, the choice for the programming language is not. In contrast, Julia combines the efficiency and numerical stability required for molecular simulations with the possibility of rapid development. \\

Several packages already exist in Julia related to structural bioinformatics. Most prominently, the packages under the two Github communities \textit{Molecular Simulation in Julia} and \textit{BioJulia}, which puts an emphasis on sequential bioinformatics \cite{JuliaMolSim, BioJulia}. 

\textit{Molly.jl} is an excellent package for molecular simulations written in Julia and is part of the \textit{Molecular Simulation in Julia} Github community \cite{Greener2024}. 
Additionally, \textit{ProtoSyn.jl} is an interesting approach to handle and manipulate oligopeptides but does not seem to be actively maintained any more (the last push is 10 months ago).

A platform from which molecular file formats can be read and write, the entire preprocessing pipeline can be integrated and the infrastructure for molecular mechanics are provided is still lacking. 
There remains a need for a basis from which software packages for handling molecular file formats and the proper preprocessing


To the best of our knowledge a comparable package for molecular analysis does not exist in Julia. Furthermore, the ongoing developments around the \ball project including the molecular viewer indicate a strong need for such a framework. 


Here, we present \biochem. We provide the basis for analysis studies encompassing the entire molecular modeling pipeline:
\begin{itemize}
	\item Reading common data formats such as PDB, hin, mol and JSON
	\item Preprocessing the input by preparing the entire system ready to simulate.
	\item Molecular Mechanics such as AMBER ForceField
	\item (Output writing) such as JSON
\end{itemize}

\biochem is designed to be a platform from which other packages can be included.
